

 The {\bfseries Catalytic Reactions Network Stochastic Simulator (Ca\-R\-Ne\-S\-S)} is a computational model devoted to the simulation of theoretical complex catalytic networks composed of different interacting molecular species. The model takes inspiration from the original model proposed by Kauffman in 1986, and describes systems composed of molecular species interacting by means of two possible reactions only, cleavage and condensation. One polymer is divided into two short polymers in the former case while two polymers are glued together forming a longer polymer in the latter case. Each reaction must be catalyzed by another species in the system to occur, and one of the assumptions is that any chemical has an independent probability to catalyze a randomly chosen reaction. It is important to notice that there are not indications about the chemical nature of the molecules, species \char`\"{}\-A\char`\"{} may be both a polipeptide, an amminoacid, a particular protein domain or an R\-N\-A strenght.\par
\par
 \hypertarget{intro_secUsage}{}\section{Using the simulator}\label{intro_secUsage}
To run the simulator open a terminal shell and type\-:\par
\par
 {\ttfamily } \$path/executive\-File {\ttfamily } $<$configuration\-\_\-\-File\-\_\-\-Folder$>$ {\ttfamily } $<$output\-\_\-folder$>$ {\ttfamily } $<$reaction\-\_\-structures\-\_\-folder$>$\par
 Examples\-:
\begin{DoxyItemize}
\item Unix Based Systems\-:{\ttfamily $\sim$/\-Documents/project/acsm2s} {\ttfamily $\sim$/\-Documents/}.../conf\-File\-Folder/ {\ttfamily $\sim$/\-Documents/}.../res\-Folder/ {\ttfamily $\sim$/\-Documents/}.../\-Structures\-Folder/
\item Win Systems\-: {\ttfamily C\-:\textbackslash{}Documents\textbackslash{}project\textbackslash{}acsm2s.\-exe} {\ttfamily C\-:\textbackslash{}Documents} ..\textbackslash{}conf\-File\-Folder\textbackslash{} {\ttfamily C\-:\textbackslash{}Documents} ..\textbackslash{}res\-Folder\textbackslash{} {\ttfamily C\-:\textbackslash{}Documents} ..\textbackslash{}Structures\-Folder\textbackslash{}
\end{DoxyItemize}

\par
\par
 \hypertarget{intro_sysreq}{}\section{System Requirement}\label{intro_sysreq}


 In order to have the simulator run correctly the recommended staff is reported\-:
\begin{DoxyItemize}
\item Mac\-Os\-X 10.\-4 or later, Linux (or in general a system U\-N\-I\-X based) or Windows O\-S (tests have been performed on Win7 and win Vista) as well
\item Q\-T4 library installed (you can download them from \href{http://qt.nokia.com/downloads,}{\tt http\-://qt.\-nokia.\-com/downloads,} the download is a little bit large)
\item G\-C\-C compiler, or similar, installed (if you need to compile the software on your machine) \par
\par
 
\end{DoxyItemize}\hypertarget{intro_ide}{}\section{I\-D\-Es}\label{intro_ide}


 The Q\-T package contains a very useful and powerful I\-D\-E called Q\-T creator in which you can compile and develop your code. Alternatively on Mac Os sytems you can use x\-Code (\href{http://developer.apple.com/xcode/}{\tt http\-://developer.\-apple.\-com/xcode/}). To create a valid x\-Code project it is sufficient, once you have installed the Q\-T libraries, to open the terminal, go into the source code folder and type \char`\"{}qmake -\/spec macx-\/xcode Q\-T\-\_\-acs.\-pro\char`\"{}.\par
 {\bfseries A\-T\-T\-E\-N\-T\-I\-O\-N!!! A version of the Q\-T libraries specific for Max\-Os\-X Lion (10.\-7) has not been yet released so it is not possible to create a valid x\-Code project, nonetheless you can always work on code by means of Q\-T creator software paying attention to add the line \char`\"{}\-Q\-M\-A\-K\-E\-\_\-\-M\-A\-C\-\_\-\-S\-D\-K = /\-Developer/\-S\-D\-Ks/\-Mac\-O\-S\-X10.\-6.\-sdk\char`\"{} in the Q\-T\-\_\-acs.\-pro file} \par
\par
 \hypertarget{intro_parameters}{}\section{Input Parameters \-:: acsm2s.\-conf}\label{intro_parameters}


 All the system parameters are stored in a file called {\bfseries acsm2s.\-conf}. Anyone can create his own configuration file paying attention to put \char`\"{}=\char`\"{} char between the parameter name and the the parameter value (N\-O S\-P\-A\-C\-E B\-E\-T\-W\-E\-E\-N T\-H\-E\-M).\par
 Notice that the simulator does not create the initial structures but it simply loads the structures created by an external software and process them. Nevertheless the configuration file is fundamental to supply all the parameters to the simulation (during the simulation new entities may be created). The simulator is provided with a structures initializator developed in M\-A\-T\-L\-A\-B language by the group (a description of the initializator is provided in the main file \char`\"{}start.\-m\char`\"{}) in which all the parameters we are going to describe are used to create the initial structures. All parameters are reported below divided in three categories\-: \begin{DoxyVerb}          - System
          - Environment
          - Dynamic

          Categories are useful only to help users in the parameter recognition within the configuration file. They are not handled from the software, if you like you can rearrange configuration file as you prefer, notice only that comments have to start with character <i>"#"</i>. Within the source code folder an example of the acsm2s.conf file is provided.<br>
 The following parameters are used both by the initializator and the simulator. Nvertheless it is ALWAYS necessary having a complete configuration file even if the structures have been already created.
          @subsection paramsystem System
          @param nGen (> 0) Number of generations. This parameter indicate how many times the simulation is stopped, concentration are set to the initial ones and the simulazion restart for other nSeconds seconds.
          @param nSIM (> 0) Number of simulations per generation starting with the same initial conditions (same data structures) but different random seed
          @param nSeconds (> 0) Number of seconds
          @param nReactions (> 0) Max number of reactions (the system will be stopped after nSeconds or after nReactions)
          @param randomSeed (>= 0) Random seed (if 0 the random seed is randomly created and the it is stored in the acsm2s.conf file saved in the results folder)
          @param nHours (>=0) Runtime limit (hours)
 @param nAttempts (>=0) Number of temptative in simulating the same network structure different random seed
 @param debugLevel (>= -1) Debug Level Runtime: different runTime message amounts (from -1 to 4, 0 is suggested)
          @param timeStructuresSavingInterval (> 0) All system structures (species, catalysis and reactions) are saved every <i>timeStructuresSavingInterval</i> seconds (simulation time)
 @param fileTimesSaveInterval (>= 0) Times data are stored in file times.csv every <i>fileTimesSaveInterval</i> seconds (If 0 reactions are stored continually)
          @subsection paramenv Environment
          @param lastFiringDiskSpeciesID (> 0) The ID of the last firing disk species.
          @param overallConcentration (> 0) The overall initial concentration that will be divided between all the initial species according to the selected initial distribution.
          @param ECConcentration (> 0) Incoming concentration of charged molecules per second.
          @param alphabet (string) Alphabet used in the simulation (e.g. <i>AGCT</i> for DNA, <i>ADEGFLYCWPHQIMTNKSRV</i> for proteins)
          @param volume (> 0) Volume of the container or protocell
          @subsection paramdyn Dynamic
          @param energy (0 or 1) 0 no energy in the system, 1 energy constraints are applied
 @param ratioSpeciesEnergizable (%) The probability for a species to be potentially energized by the energy carriers
          @param complexFormationSymmetry (0 or 1) Complex Formation Symmetry
                          - <b>1</b>: the catalyst can bound both substrates
                          - <b>0 (default)</b>: catalyst binds only with the first substrate of the reaction
          @param nonCatalyticMaxLength (>= 0) Max length of non catalytic species
          @param reactionProbability (from 0 to 1) Probability for a species to catalyze a reaction
          @param cleavageProbability (from 0 to 1) Cleavage probability (Condensation probability is 1 - cleavage probability)
          @param reverseReaction (0 or 1) Set to 1 to enable reverse reactions, 0 otherwise
 @param revRctRatio (>0) Ratio between forward and backward reactions, it is used in the creation of new reactions only (if reverseReactions = TRUE)
          @param K_ass (> 0) Final Condensation kinetic constant (C.A + B --> AB + C) where A.C is the molecular complex composed of C (the catalyst) and A (the first substrate)
          @param K_diss (> 0) Cleavage kinetic constant (AB --> A + B)
          @param K_cpx (> 0) Complex formation kinetic constant (A + C(catalyst) --> C.A)
          @param K_cpxDiss (> 0) Complex Dissociation kinetic constant (C.A --> A + C)
          @param K_nrg (> 0) species phosphorilation kinetic constant
          @param K_nrg_decay (> 0) de-energization kinetic constant
          @param moleculeDecay_KineticConstant (> 0) Molecule decay (efflux) kinetic Constant (Disregarded if the system is closed)
          @param influx_rate (>= 0) Concentration per seconds (The species to insert in the system will be randomly chosen according to the _acsinflux.csv file). If equal to 0 the system is closed (maxLOut=0) or only the species that can cross the membrane come in and go out (maxLOut>0).
          @param maxLOut Maximum lenght of the species involved in the efflux process (\c influx_rate  > 0), equal to 0 indicates that all the species can be involved in the efflux process (no filter). If influx_rate = 0 the parameter indicates the species that can cross the semipermeable membrane of the protocell. <b>THE COUPLING BETWEEN INFLUX_RATE AND MAXLOUT INDICATES IF WE ARE SIMULATING A PROTOCELL OR A FLOW REACTOR</b>
          @param diffusion_contribute (KD) (0 or 0.5) if set to 0.5 the speed of molecules goes with the inverse of the square of the length, L^{-KD}
          @param solubility_threshold (> 0) Solubility Threshold, all the species longer than solubility_threshold precipitate
\end{DoxyVerb}


\par
\par
 \hypertarget{intro_Aknowledgments}{}\section{Aknowledgments}\label{intro_Aknowledgments}



\begin{DoxyItemize}
\item University of Bologna, Interdepartment of industrial research (C.\-I.\-R.\-I)
\item European Centre for Living Technology \href{http://www.ecltech.org/}{\tt http\-://www.\-ecltech.\-org/}
\item Fondazione Venezia \href{http://www.fondazionevenezia.it}{\tt http\-://www.\-fondazionevenezia.\-it}
\item Roberto Serra, Marco Villani, Timoteo Carletti, Alex Graudenzi, Norman Packard, Ruedi Fuchslin and Stuart Kauffman for the essential hints.
\item \href{http://www.bedaux.net/mtrand/}{\tt http\-://www.\-bedaux.\-net/mtrand/} for the pseudo-\/random Marseinne-\/\-Twister library for C++.
\item \href{http://perso.wanadoo.es/antlarr/otherapps.html}{\tt http\-://perso.\-wanadoo.\-es/antlarr/otherapps.\-html} for the poisson distribution generator numbers (acs\-\_\-long\-Int \hyperlink{common_functions_8h_a22cddb6ffcf2250e0c90bc913728350f}{random\-\_\-poisson(acs\-\_\-double tmp\-Lambda, M\-T\-Rand\& tmp\-Random\-Generator)}).
\item Dr. Luca Ansaloni (\href{mailto:luca.ansaloni@unimore.it}{\tt luca.\-ansaloni@unimore.\-it}) for the support but especially for the file handling functions and new Python development. 
\end{DoxyItemize}