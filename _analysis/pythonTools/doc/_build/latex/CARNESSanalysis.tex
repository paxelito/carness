% Generated by Sphinx.
\def\sphinxdocclass{report}
\documentclass[letterpaper,10pt,english]{sphinxmanual}
\usepackage[utf8]{inputenc}
\DeclareUnicodeCharacter{00A0}{\nobreakspace}
\usepackage{cmap}
\usepackage[T1]{fontenc}
\usepackage{babel}
\usepackage{times}
\usepackage[Bjarne]{fncychap}
\usepackage{longtable}
\usepackage{sphinx}
\usepackage{multirow}


\title{CARNESS analysis package}
\date{April 24, 2015}
\release{20150424.001}
\author{Alessandro Filisetti}
\newcommand{\sphinxlogo}{\includegraphics{logo.png}\par}
\renewcommand{\releasename}{Release}
\makeindex

\makeatletter
\def\PYG@reset{\let\PYG@it=\relax \let\PYG@bf=\relax%
    \let\PYG@ul=\relax \let\PYG@tc=\relax%
    \let\PYG@bc=\relax \let\PYG@ff=\relax}
\def\PYG@tok#1{\csname PYG@tok@#1\endcsname}
\def\PYG@toks#1+{\ifx\relax#1\empty\else%
    \PYG@tok{#1}\expandafter\PYG@toks\fi}
\def\PYG@do#1{\PYG@bc{\PYG@tc{\PYG@ul{%
    \PYG@it{\PYG@bf{\PYG@ff{#1}}}}}}}
\def\PYG#1#2{\PYG@reset\PYG@toks#1+\relax+\PYG@do{#2}}

\expandafter\def\csname PYG@tok@gd\endcsname{\def\PYG@tc##1{\textcolor[rgb]{0.63,0.00,0.00}{##1}}}
\expandafter\def\csname PYG@tok@gu\endcsname{\let\PYG@bf=\textbf\def\PYG@tc##1{\textcolor[rgb]{0.50,0.00,0.50}{##1}}}
\expandafter\def\csname PYG@tok@gt\endcsname{\def\PYG@tc##1{\textcolor[rgb]{0.00,0.27,0.87}{##1}}}
\expandafter\def\csname PYG@tok@gs\endcsname{\let\PYG@bf=\textbf}
\expandafter\def\csname PYG@tok@gr\endcsname{\def\PYG@tc##1{\textcolor[rgb]{1.00,0.00,0.00}{##1}}}
\expandafter\def\csname PYG@tok@cm\endcsname{\let\PYG@it=\textit\def\PYG@tc##1{\textcolor[rgb]{0.25,0.50,0.56}{##1}}}
\expandafter\def\csname PYG@tok@vg\endcsname{\def\PYG@tc##1{\textcolor[rgb]{0.73,0.38,0.84}{##1}}}
\expandafter\def\csname PYG@tok@m\endcsname{\def\PYG@tc##1{\textcolor[rgb]{0.13,0.50,0.31}{##1}}}
\expandafter\def\csname PYG@tok@mh\endcsname{\def\PYG@tc##1{\textcolor[rgb]{0.13,0.50,0.31}{##1}}}
\expandafter\def\csname PYG@tok@cs\endcsname{\def\PYG@tc##1{\textcolor[rgb]{0.25,0.50,0.56}{##1}}\def\PYG@bc##1{\setlength{\fboxsep}{0pt}\colorbox[rgb]{1.00,0.94,0.94}{\strut ##1}}}
\expandafter\def\csname PYG@tok@ge\endcsname{\let\PYG@it=\textit}
\expandafter\def\csname PYG@tok@vc\endcsname{\def\PYG@tc##1{\textcolor[rgb]{0.73,0.38,0.84}{##1}}}
\expandafter\def\csname PYG@tok@il\endcsname{\def\PYG@tc##1{\textcolor[rgb]{0.13,0.50,0.31}{##1}}}
\expandafter\def\csname PYG@tok@go\endcsname{\def\PYG@tc##1{\textcolor[rgb]{0.20,0.20,0.20}{##1}}}
\expandafter\def\csname PYG@tok@cp\endcsname{\def\PYG@tc##1{\textcolor[rgb]{0.00,0.44,0.13}{##1}}}
\expandafter\def\csname PYG@tok@gi\endcsname{\def\PYG@tc##1{\textcolor[rgb]{0.00,0.63,0.00}{##1}}}
\expandafter\def\csname PYG@tok@gh\endcsname{\let\PYG@bf=\textbf\def\PYG@tc##1{\textcolor[rgb]{0.00,0.00,0.50}{##1}}}
\expandafter\def\csname PYG@tok@ni\endcsname{\let\PYG@bf=\textbf\def\PYG@tc##1{\textcolor[rgb]{0.84,0.33,0.22}{##1}}}
\expandafter\def\csname PYG@tok@nl\endcsname{\let\PYG@bf=\textbf\def\PYG@tc##1{\textcolor[rgb]{0.00,0.13,0.44}{##1}}}
\expandafter\def\csname PYG@tok@nn\endcsname{\let\PYG@bf=\textbf\def\PYG@tc##1{\textcolor[rgb]{0.05,0.52,0.71}{##1}}}
\expandafter\def\csname PYG@tok@no\endcsname{\def\PYG@tc##1{\textcolor[rgb]{0.38,0.68,0.84}{##1}}}
\expandafter\def\csname PYG@tok@na\endcsname{\def\PYG@tc##1{\textcolor[rgb]{0.25,0.44,0.63}{##1}}}
\expandafter\def\csname PYG@tok@nb\endcsname{\def\PYG@tc##1{\textcolor[rgb]{0.00,0.44,0.13}{##1}}}
\expandafter\def\csname PYG@tok@nc\endcsname{\let\PYG@bf=\textbf\def\PYG@tc##1{\textcolor[rgb]{0.05,0.52,0.71}{##1}}}
\expandafter\def\csname PYG@tok@nd\endcsname{\let\PYG@bf=\textbf\def\PYG@tc##1{\textcolor[rgb]{0.33,0.33,0.33}{##1}}}
\expandafter\def\csname PYG@tok@ne\endcsname{\def\PYG@tc##1{\textcolor[rgb]{0.00,0.44,0.13}{##1}}}
\expandafter\def\csname PYG@tok@nf\endcsname{\def\PYG@tc##1{\textcolor[rgb]{0.02,0.16,0.49}{##1}}}
\expandafter\def\csname PYG@tok@si\endcsname{\let\PYG@it=\textit\def\PYG@tc##1{\textcolor[rgb]{0.44,0.63,0.82}{##1}}}
\expandafter\def\csname PYG@tok@s2\endcsname{\def\PYG@tc##1{\textcolor[rgb]{0.25,0.44,0.63}{##1}}}
\expandafter\def\csname PYG@tok@vi\endcsname{\def\PYG@tc##1{\textcolor[rgb]{0.73,0.38,0.84}{##1}}}
\expandafter\def\csname PYG@tok@nt\endcsname{\let\PYG@bf=\textbf\def\PYG@tc##1{\textcolor[rgb]{0.02,0.16,0.45}{##1}}}
\expandafter\def\csname PYG@tok@nv\endcsname{\def\PYG@tc##1{\textcolor[rgb]{0.73,0.38,0.84}{##1}}}
\expandafter\def\csname PYG@tok@s1\endcsname{\def\PYG@tc##1{\textcolor[rgb]{0.25,0.44,0.63}{##1}}}
\expandafter\def\csname PYG@tok@gp\endcsname{\let\PYG@bf=\textbf\def\PYG@tc##1{\textcolor[rgb]{0.78,0.36,0.04}{##1}}}
\expandafter\def\csname PYG@tok@sh\endcsname{\def\PYG@tc##1{\textcolor[rgb]{0.25,0.44,0.63}{##1}}}
\expandafter\def\csname PYG@tok@ow\endcsname{\let\PYG@bf=\textbf\def\PYG@tc##1{\textcolor[rgb]{0.00,0.44,0.13}{##1}}}
\expandafter\def\csname PYG@tok@sx\endcsname{\def\PYG@tc##1{\textcolor[rgb]{0.78,0.36,0.04}{##1}}}
\expandafter\def\csname PYG@tok@bp\endcsname{\def\PYG@tc##1{\textcolor[rgb]{0.00,0.44,0.13}{##1}}}
\expandafter\def\csname PYG@tok@c1\endcsname{\let\PYG@it=\textit\def\PYG@tc##1{\textcolor[rgb]{0.25,0.50,0.56}{##1}}}
\expandafter\def\csname PYG@tok@kc\endcsname{\let\PYG@bf=\textbf\def\PYG@tc##1{\textcolor[rgb]{0.00,0.44,0.13}{##1}}}
\expandafter\def\csname PYG@tok@c\endcsname{\let\PYG@it=\textit\def\PYG@tc##1{\textcolor[rgb]{0.25,0.50,0.56}{##1}}}
\expandafter\def\csname PYG@tok@mf\endcsname{\def\PYG@tc##1{\textcolor[rgb]{0.13,0.50,0.31}{##1}}}
\expandafter\def\csname PYG@tok@err\endcsname{\def\PYG@bc##1{\setlength{\fboxsep}{0pt}\fcolorbox[rgb]{1.00,0.00,0.00}{1,1,1}{\strut ##1}}}
\expandafter\def\csname PYG@tok@kd\endcsname{\let\PYG@bf=\textbf\def\PYG@tc##1{\textcolor[rgb]{0.00,0.44,0.13}{##1}}}
\expandafter\def\csname PYG@tok@ss\endcsname{\def\PYG@tc##1{\textcolor[rgb]{0.32,0.47,0.09}{##1}}}
\expandafter\def\csname PYG@tok@sr\endcsname{\def\PYG@tc##1{\textcolor[rgb]{0.14,0.33,0.53}{##1}}}
\expandafter\def\csname PYG@tok@mo\endcsname{\def\PYG@tc##1{\textcolor[rgb]{0.13,0.50,0.31}{##1}}}
\expandafter\def\csname PYG@tok@mi\endcsname{\def\PYG@tc##1{\textcolor[rgb]{0.13,0.50,0.31}{##1}}}
\expandafter\def\csname PYG@tok@kn\endcsname{\let\PYG@bf=\textbf\def\PYG@tc##1{\textcolor[rgb]{0.00,0.44,0.13}{##1}}}
\expandafter\def\csname PYG@tok@o\endcsname{\def\PYG@tc##1{\textcolor[rgb]{0.40,0.40,0.40}{##1}}}
\expandafter\def\csname PYG@tok@kr\endcsname{\let\PYG@bf=\textbf\def\PYG@tc##1{\textcolor[rgb]{0.00,0.44,0.13}{##1}}}
\expandafter\def\csname PYG@tok@s\endcsname{\def\PYG@tc##1{\textcolor[rgb]{0.25,0.44,0.63}{##1}}}
\expandafter\def\csname PYG@tok@kp\endcsname{\def\PYG@tc##1{\textcolor[rgb]{0.00,0.44,0.13}{##1}}}
\expandafter\def\csname PYG@tok@w\endcsname{\def\PYG@tc##1{\textcolor[rgb]{0.73,0.73,0.73}{##1}}}
\expandafter\def\csname PYG@tok@kt\endcsname{\def\PYG@tc##1{\textcolor[rgb]{0.56,0.13,0.00}{##1}}}
\expandafter\def\csname PYG@tok@sc\endcsname{\def\PYG@tc##1{\textcolor[rgb]{0.25,0.44,0.63}{##1}}}
\expandafter\def\csname PYG@tok@sb\endcsname{\def\PYG@tc##1{\textcolor[rgb]{0.25,0.44,0.63}{##1}}}
\expandafter\def\csname PYG@tok@k\endcsname{\let\PYG@bf=\textbf\def\PYG@tc##1{\textcolor[rgb]{0.00,0.44,0.13}{##1}}}
\expandafter\def\csname PYG@tok@se\endcsname{\let\PYG@bf=\textbf\def\PYG@tc##1{\textcolor[rgb]{0.25,0.44,0.63}{##1}}}
\expandafter\def\csname PYG@tok@sd\endcsname{\let\PYG@it=\textit\def\PYG@tc##1{\textcolor[rgb]{0.25,0.44,0.63}{##1}}}

\def\PYGZbs{\char`\\}
\def\PYGZus{\char`\_}
\def\PYGZob{\char`\{}
\def\PYGZcb{\char`\}}
\def\PYGZca{\char`\^}
\def\PYGZam{\char`\&}
\def\PYGZlt{\char`\<}
\def\PYGZgt{\char`\>}
\def\PYGZsh{\char`\#}
\def\PYGZpc{\char`\%}
\def\PYGZdl{\char`\$}
\def\PYGZhy{\char`\-}
\def\PYGZsq{\char`\'}
\def\PYGZdq{\char`\"}
\def\PYGZti{\char`\~}
% for compatibility with earlier versions
\def\PYGZat{@}
\def\PYGZlb{[}
\def\PYGZrb{]}
\makeatother

\begin{document}

\maketitle
\tableofcontents
\phantomsection\label{index::doc}


The CARNESS analysis python package contains a set of python scripts devoted to
\begin{enumerate}
\item {} 
the creation and the structural analysis of artificial catalytic reaction networks (CRS).

\item {} 
the analysis of the dynamical outcomes of the artificial catalytic reaction networks simulated by means of the CAtalytic Reaction NEtworks Stochastic Simulator (\href{https://github.com/paxelito/carness}{CARNESS}).

\end{enumerate}


\chapter{System Requirements}
\label{index:carness-analysis-python-package-documentation}\label{index:system-requirements}
In order to use the scripts contained in this package, some further packages need to be installed.
On UNIX systems you can install the necessary packages by the usual commands, for example on DEBIAN-based systems:

\begin{Verbatim}[commandchars=\\\{\}]
\PYGZdl{} sudo apt\PYGZhy{}get update
\PYGZdl{} sudo apt\PYGZhy{}get install python\PYGZhy{}numpy python\PYGZhy{}scipy python\PYGZhy{}matplotlib python\PYGZhy{}setuptools python\PYGZhy{}networkx python\PYGZhy{}deap graphviz libgraphviz\PYGZhy{}dev pkg\PYGZhy{}config python\PYGZhy{}pygraphviz
\end{Verbatim}

On WINDOWS system, I personally suggest that the simplest solution may be to rely on the alternative solutions proposed \href{https://www.python.org/download/alternatives/}{here}.


\chapter{Installing}
\label{index:installing}
To get the packages (the package is entirely contained in a folder named ``ACS\_analysis'') it is sufficent to clone it from GIT with the command:

\begin{Verbatim}[commandchars=\\\{\}]
git clone https://github.com/paxelito/ACS\PYGZus{}analysis.git
\end{Verbatim}

Since the package is still under development it is a good practice to check for possible updates or upgrades each time you use it. To seach for new updates please run the following command within the ``ACS\_analysis'' folder

\begin{Verbatim}[commandchars=\\\{\}]
git pull
\end{Verbatim}


\chapter{Usage}
\label{index:usage}
To use the package you must run the opportune python script with the appropriate parameters. You can find a detailed description of the available scripts within this documentation.
It is important to notice that not all the actually available scripts are depicted in this documentation. Hence, if you are familiar with python and with the CARNESS simulator, you can search into the package whether the desired analysis have been already implemented.
Out objective is that of preparing the documentation for all the packages in the next few months.

Current Available Scripts Analysis:


\section{initializator Module}
\label{initializator:initializator-module}\label{initializator::doc}\label{initializator:module-initializator}\index{initializator (module)}
This script concerns all the aspect of the creation and the initialization of artificial catalytic reaction networks in th format 
requested by the \href{http://github.org/carness}{CARNESS simulation platform}.
To have a description of all the parameters admitted by the initializator plase digit:

\begin{Verbatim}[commandchars=\\\{\}]
python \PYGZlt{}path\PYGZgt{}/initializator.py \PYGZhy{}h
\end{Verbatim}


\subsection{TO Do}
\label{initializator:to-do}\begin{itemize}
\item {} 
Introduce here all the parameters allow to upload all the parameters from file, not some here and some on file

\end{itemize}


\section{Chemistry Graph Analysis}
\label{graph_chemistry_analysis:chemistry-graph-analysis}\label{graph_chemistry_analysis:module-graph_chemistry_analysis}\label{graph_chemistry_analysis::doc}\index{graph\_chemistry\_analysis (module)}
This script evaluates a selected chemistry finding RAF, SCC and 
saving the bipartite multigraph and the catalyst-product network. 
RAF and SCC summaries are saved too.

Script SYNOPSIS:

usage: graph\_chemistry\_analysis.py {[}-h{]} {[}-p STRPATH{]} {[}-f LASTFLUXID{]} {[}-m MAXDIM{]}

Graph analysis of the chemistry
\begin{description}
\item[{optional arguments:}] \leavevmode\begin{optionlist}{3cm}
\item [-h, -{-}help]  
show this help message and exit
\item [-p STRPATH, -{-}strPath STRPATH]  
Path of the artificial chemistry to analyze (def: ./)
\item [-f LASTFLUXID, -{-}lastFluxID LASTFLUXID]  
Last ID of the food species (def: 5)
\item [-m MAXDIM, -{-}maxDim MAXDIM]  
Max Dimension of the system (def: 6)
\end{optionlist}

\end{description}


\subsection{Bipartite MultiGraph legend}
\label{graph_chemistry_analysis:bipartite-multigraph-legend}\begin{itemize}
\item {} \begin{description}
\item[{Nodes}] \leavevmode\begin{itemize}
\item {} 
Red Circle: Molecular species

\item {} 
Blue Circle: Molecular species belonging to the SCCs (only in ALL reactions representations)

\item {} 
Green Square: reactions

\end{itemize}

\end{description}

\item {} \begin{description}
\item[{Edges}] \leavevmode\begin{itemize}
\item {} 
Grey: Substrate or product, according to the direction of the arrow, partipation

\item {} 
Blue: Catalysis

\item {} 
Red: WARNING arrow. It means that a species is both a catalyst and a substrate of the reaction.

\end{itemize}

\end{description}

\end{itemize}


\subsection{OUTPUT files}
\label{graph_chemistry_analysis:output-files}
Output are stored in a folder named \_0\_new\_allStatResults located within the chemistry directory. 
Files list:
\begin{itemize}
\item {} 
completebipartitegraph.png.{[}png/net{]} :: Bipartite Multigraph of all the chemistry

\item {} 
bipartiteRAFgraph.png.{[}png/net{]} :: Bipartite Multigraph of the reactions involved in the RAF only

\item {} 
chemistry\_cat\_prod\_graph.{[}png/net{]} :: Catalyst -\textgreater{} Product representation of the chemistry

\item {} 
0\_initRafAnalysis.csv :: Summary of the RAF properties

\item {} 
0\_initRafAnalysisALL :: List of the reactions composing the RAF

\item {} 
0\_initRafAnalysisLIST :: Enumeration of the species and reactions composing the RAF

\end{itemize}

Currently graphs are exported in PAJEK (\href{http://mrvar.fdv.uni-lj.si/pajek/}{http://mrvar.fdv.uni-lj.si/pajek/}) format, other formats are available at \href{http://networkx.lanl.gov/reference/readwrite.html}{http://networkx.lanl.gov/reference/readwrite.html}


\section{Protocell Duplication Analysis}
\label{acsDuplicationAnalysis:protocell-duplication-analysis}\label{acsDuplicationAnalysis:module-acsDuplicationAnalysis}\label{acsDuplicationAnalysis::doc}\index{acsDuplicationAnalysis (module)}
Script to compute the successive cell division times and the value of each molecule in proximity of the cell division.
Please digit:

\begin{Verbatim}[commandchars=\\\{\}]
python \PYGZlt{}path\PYGZgt{}/acsDuplicationAnalysis.py
\end{Verbatim}

for the SYNOPSIS of the script.

In particular by means of this analysis three files are saved:
\begin{enumerate}
\item {} 
deltat\_\textless{}CHEMISTRY\textgreater{}.csv: In this file the cell division time and the overall amount of each molecular species at each division are stored

\item {} 
delta\_ALL\_\textless{}CHEMISTRY\textgreater{}.csv: In this file the overall amount of each molecular species at each division is stored

\item {} 
divplot\_\textless{}CHEMISTRY\textgreater{}.{[}png/eps{]}: If \emph{param --graphs -g}, hence the plot of the species amount at each generation is generated.

\end{enumerate}


\chapter{Indices}
\label{index:indices}\begin{itemize}
\item {} 
\emph{genindex}

\item {} 
\emph{modindex}

\item {} 
\emph{search}

\end{itemize}


\renewcommand{\indexname}{Python Module Index}
\begin{theindex}
\def\bigletter#1{{\Large\sffamily#1}\nopagebreak\vspace{1mm}}
\bigletter{a}
\item {\texttt{acsDuplicationAnalysis}}, \pageref{acsDuplicationAnalysis:module-acsDuplicationAnalysis}
\indexspace
\bigletter{g}
\item {\texttt{graph\_chemistry\_analysis}}, \pageref{graph_chemistry_analysis:module-graph_chemistry_analysis}
\indexspace
\bigletter{i}
\item {\texttt{initializator}}, \pageref{initializator:module-initializator}
\end{theindex}

\renewcommand{\indexname}{Index}
\printindex
\end{document}
